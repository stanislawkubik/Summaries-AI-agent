\documentclass[11pt]{article}

\usepackage[margin=1in]{geometry}
\usepackage{amsmath}
\usepackage{amssymb}
\usepackage{hyperref}

% Compact, referenceable key-results map.
\newenvironment{keyresultmap}{\begin{enumerate}}{\end{enumerate}}

\title{[[CONCEPT]]}
\author{}
\date{}

\begin{document}
\maketitle

\section*{Prerequisite concepts}
% One line or a short list of prerequisites with specific focus areas.
[[PREREQUISITES_LINE]]

\section*{What you will learn}
% 3--6 bullet learning objectives with action verbs.
\begin{itemize}
  \item [[LEARNING_OBJECTIVE_1]]
  \item [[LEARNING_OBJECTIVE_2]]
  \item [[LEARNING_OBJECTIVE_3]]
\end{itemize}

\section*{Notation and conventions}
% Define every symbol exactly once; do not redefine later.
\subsection*{Notation}
% List symbols, shapes, and meanings; define all acronyms on first use.
[[NOTATION_BLOCK]]

\subsection*{Conventions}
% Force explicit choices about data preprocessing, intercepts, and scaling.
\begin{itemize}
  \item Data shapes and symbol meanings: [[DATA_SHAPES_AND_MEANINGS]]
  \item Centering and standardizing conventions: [[CENTERING_STANDARDIZATION]]
  \item Intercept treatment: [[INTERCEPT_TREATMENT]]
  \item Objective scaling conventions: [[SCALING_CONVENTIONS]]
  \item Mapping to common library parameter names (if relevant): [[LIBRARY_PARAMETER_MAPPING]]
\end{itemize}

\section{Problem setup and motivation}
% Describe the concrete problem, why it matters, and what fails without this concept.
[[PROBLEM_AND_MOTIVATION]]

\section{General idea}
% One cohesive paragraph: what it is, why it works at a high level, and when it is used.
[[GENERAL_IDEA]]

\section{Intuition}
% Provide at least two lenses chosen from: geometric view, spectral or decomposition view,
% probabilistic or Bayesian view, optimization view, toy example, limiting cases.
% Each lens must end with a one-sentence takeaway.
\subsection*{Lens 1: [[INTUITION_LENS_1]]}
[[INTUITION_TEXT_1]]
\textbf{Takeaway:} [[TAKEAWAY_1]]

\subsection*{Lens 2: [[INTUITION_LENS_2]]}
[[INTUITION_TEXT_2]]
\textbf{Takeaway:} [[TAKEAWAY_2]]

% Optional third lens if helpful.
% \subsection*{Lens 3: [[INTUITION_LENS_3]]}
% [[INTUITION_TEXT_3]]
% \textbf{Takeaway:} [[TAKEAWAY_3]]

\section{Formal definition}
% Precisely define variables, assumptions, and conventions. Resolve scaling/standardization/intercept ambiguities.
% Every equation must have an explanatory sentence stating what it means and why it is included.
Let [[FORMAL_SYMBOL_DEFINITIONS]].
\begin{equation}
  \label{eq:core-1}
  [[EQUATION_1]]
\end{equation}
[[EQUATION_1_EXPLANATION]]

\begin{equation}
  \label{eq:core-2}
  [[EQUATION_2]]
\end{equation}
[[EQUATION_2_EXPLANATION]]

% Add additional equations as needed, each with a clear explanation sentence.

\section{Key results map}
% 3--7 results. Each item must point to where it is derived (section or appendix label).
\begin{keyresultmap}
  \item \textbf{[[RESULT_1_TITLE]]}\label{kr:result-1} [[RESULT_1_STATEMENT]] (Derived in Section~\ref{sec:result-1} or Appendix~\ref{app:result-1}.)
  \item \textbf{[[RESULT_2_TITLE]]}\label{kr:result-2} [[RESULT_2_STATEMENT]] (Derived in Section~\ref{sec:result-2} or Appendix~\ref{app:result-2}.)
  \item \textbf{[[RESULT_3_TITLE]]}\label{kr:result-3} [[RESULT_3_STATEMENT]] (Derived in Section~\ref{sec:result-3} or Appendix~\ref{app:result-3}.)
\end{keyresultmap}

\section{Estimation, tuning, and computation}
% Describe estimation procedures, tuning choices, algorithmic complexity, and practical implementation tips.
[[ESTIMATION_TUNING_COMPUTATION]]

\section{Diagnostics and interpretation}
% Explain how to check fit, validate assumptions, and interpret parameters or outputs.
[[DIAGNOSTICS_AND_INTERPRETATION]]

\section{Common confusions and failure modes}
% Format each item as: symptom, cause, fix.
\begin{itemize}
  \item \textbf{Symptom:} [[SYMPTOM_1]] \textbf{Cause:} [[CAUSE_1]] \textbf{Fix:} [[FIX_1]]
  \item \textbf{Symptom:} [[SYMPTOM_2]] \textbf{Cause:} [[CAUSE_2]] \textbf{Fix:} [[FIX_2]]
\end{itemize}

\section{Connections and extensions}
% Relate to adjacent concepts, generalizations, or alternative formulations.
[[CONNECTIONS_AND_EXTENSIONS]]

% Optional: include only if the user requests a domain or context.
% \section{Applications}
% [[APPLICATIONS_SECTION]]

\section{Further reading}
% Keep it tight; avoid long literature surveys unless requested.
\subsection*{Foundational paper}
\begin{itemize}
  \item [[FOUNDATIONAL_PAPER]] \cite{[[FOUNDATIONAL_KEY]]}
\end{itemize}

\subsection*{Best notes or survey}
\begin{itemize}
  \item [[BEST_NOTES]] \cite{[[NOTES_KEY]]}
\end{itemize}

\subsection*{Textbook}
\begin{itemize}
  \item [[TEXTBOOK_REF]] \cite{[[TEXTBOOK_KEY]]}
\end{itemize}

\subsection*{Implementation docs}
\begin{itemize}
  \item [[IMPLEMENTATION_DOCS]] \cite{[[IMPL_KEY]]}
\end{itemize}

\renewcommand{\refname}{Bibliography}
\begin{thebibliography}{9}
\bibitem{[[FOUNDATIONAL_KEY]]}
[[FOUNDATIONAL_BIB]]

\bibitem{[[NOTES_KEY]]}
[[NOTES_BIB]]

\bibitem{[[TEXTBOOK_KEY]]}
[[TEXTBOOK_BIB]]

\bibitem{[[IMPL_KEY]]}
[[IMPL_BIB]]
\end{thebibliography}

\appendix
\section{Derivations}
% Include every derivation needed for the key results map, with explicit labels.
\subsection{[[DERIVATION_TITLE_1]]}\label{app:result-1}
[[DERIVATION_TEXT_1]]
\begin{align}
  [[DERIVATION_EQUATION_1]]
\end{align}

\subsection{[[DERIVATION_TITLE_2]]}\label{app:result-2}
[[DERIVATION_TEXT_2]]
\begin{align}
  [[DERIVATION_EQUATION_2]]
\end{align}

% Add more derivations as needed and ensure labels match the key results map.

\end{document}
